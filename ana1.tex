\documentclass[12pt,a4paper,titlepage]{article} %twopage

%Befehle

%Grafik einbinden
%\centering
%\includegraphics[width=0.7\textwidth]{Profilbild.png}


\usepackage{german} %deutsches Format
\usepackage[utf8]{inputenc} %Umlaute
\usepackage{graphicx} %Grafiken einbinden
\usepackage{nicefrac}
\usepackage{hyperref} %Hyperlinks in pdf

%Metadaten
\hypersetup{
	pdftitle = {Analysis I Skript (WS 18/19)},
	pdfauthor = {Pavel Zwerschke}}

\newcommand{\bsp}{\textbf{Beispiel:}\\}
\newcommand{\definition}{\textbf{Definition:\\}}
\newcommand{\satz}{\textbf{Satz:\\}}

\begin{document}
\title{Analysis I (WS 18/19)}
\date{\today}
\author{Pavel Zwerschke}
\maketitle

%Inhaltsverzeichnis
\tableofcontents
\newpage

\setcounter{section}{-1}
\section{Organisatorisches}
\textbf{Dozent}\\
Prof. Dr. Dirk Hundertmark (20.30, 2.028)\\
\href{mailto:dirk.hundertmark@kit.edu}{dirk.hundertmark@kit.edu}

\textbf{Übungsleiter}\\
Dr. Markus Lange (20.30, 2.030)\\
\href{mailto:markus.lange@kit.edu}{markus.lange@kit.edu}

\textbf{Übungszettel}\\
Ausgabe:\\
donnerstags unter \href{http://www.math.kit.edu/iana1/lehre/ana12018w/}{\texttt{www.math.kit.edu/iana1/lehre/ana12018w/}}\\
Abgabe:\\
bis mittwochs um 19:00 in den Abgabekästen des Foyers des Mathematikgebäudes (20.30)\\
getackert, mit Namen, Matrikelnummer, Tutoriennummer und Deckblatt (optional) in das Fach mit der richtigen Kennzeichnung legen\\
Zettel dürfen zu zweit abgegeben werden

\textbf{Übungsschein}
Jede K-Aufgabe wird mit 4 Punkten bewertet. Einen Übungsschein erhält wer 50\% der Punkte aller K-Aufgaben erzielt.

\textbf{Klausur}\\
Die Anmeldung findet über das Online-Portal statt. Die Klausur findet in KW 8 2019 statt. Der Übungsschein ist Voraussetzung für die Teilnahme an der Klausur.

\newpage

\section{Was ist Analysis?}
\textbf{Zentrale Begriffe:}\\
Grenzwerte von Folgen und Reihen, Funktionen, stetig, differenzierbar, integrieren, Differential- und Integralrechnung, Differentialgleichungen (Newton, Maxwell, Schrödinger), unendlich dimensionale Räume\\

\bsp
$S = \frac{1}{2} + \frac{1}{4} + \dots + \frac{1}{2^n} + \dots\\
2S = 1 + \frac{1}{2} + \dots + \nicefrac{1}{2} + \dots\\$
$2S = 1 + S$

\end{document}
