\documentclass[../ana1u.tex]{subfiles}
\begin{document}
\setcounter{section}{5}

\section{k-adische Brüche (30.11.18)}
\begin{bsp}
    sei \(b \in \N \quad b \geq 2 \) \\
    unter einem b-adischen Bruch verstehen wir:
    \[\pm \limes{n} \sum_{j=-k}^{n} a_j \cdot b^{-j} \]
    Dabei ist \(k \geq 0 \) und \(a_j \) sind natürliche Zahlen mit \(0 \leq a_j \leq b \) \\
    Sei die Basis b festgelegt.
    \[\pm a_{-k}, \pm a_{-k+1}, \dots ,\pm a_{-1}, \pm a_0, \pm a_1, \pm a_2, \dots \] \\
    Für \(b = 10 \) spricht man von Dezimalbrüchen, \(b = 2 \) von dyadischen Brüchen, 
    \(b = 3 \) von ternären Brüchen. \\
    In der Vorlesung wurde die b-adische Darstellung von \(0 < \alpha < 1\) eingeführt. \\
    Wir können jede positive Zahl \(x \in \R, x > 0\) schreiben als:
    \[x = {\lfloor x \rfloor}_{\in \N} + \alpha_{\in [0, 1]} \]
\end{bsp}
\begin{defi}
    Abrundungsfunktion: \\
    Für jede reelle Zahl \(x\) ist \(\lfloor x \rfloor\) die größte ganze Zahl, 
    die kleiner oder gleich \(x\) ist.
    \[\lfloor x \rfloor = max\{k \in \Z | k \leq x\} \]
    \begin{bsp}
        \(x = 25,7365 \) dann ist \(\lfloor x \rfloor = 25 \) und \(\alpha = 0,7365 \)
    \end{bsp}
\end{defi}
\begin{defi}
    Sei \(b \in \N \) größer als 1 \\
    Eine b-adische Darstellung (eine Darstellung zur Basis b) einer natürlichen Zahl \(n \) 
    ist eine Liste \(a_m, \dots, a_0 \) von natürlichen Zahlen aus der Menge \\
    \(\{0, \dots, b-1\} \), so dass \(a_m > 0 \) und \(n = \sum_{i=0}^{m} a_ib^i \)
\end{defi}
\begin{bsp}
    Ternärdarstellung (\(b = 3\)) für die natürlichen Zahlen \(\leq 10\)
    \begin{align*}
        &n = 1 = \textbf{0} \cdot 3^2 + \textbf{0} \cdot 3^1 + \textbf{1} \cdot 3^0 
        &= 1_{(3)} \\
        &n = 2 = \textbf{0} \cdot 3^2 + \textbf{0} \cdot 3^1 + \textbf{2} \cdot 3^0 
        &= 2_{(3)} \\
        &n = 3 = \textbf{0} \cdot 3^2 + \textbf{1} \cdot 3^1 + \textbf{0} \cdot 3^0 
        &= 10_{(3)} \\
        &n = 4 = \textbf{0} \cdot 3^2 + \textbf{1} \cdot 3^1 + \textbf{1} \cdot 3^0 
        &= 11_{(3)} \\
        &n = 5 = \textbf{0} \cdot 3^2 + \textbf{1} \cdot 3^1 + \textbf{2} \cdot 3^0 
        &= 12_{(3)} \\
        &n = 6 = \textbf{0} \cdot 3^2 + \textbf{2} \cdot 3^1 + \textbf{0} \cdot 3^0 
        &= 20_{(3)} \\
        &n = 7 = \textbf{0} \cdot 3^2 + \textbf{2} \cdot 3^1 + \textbf{1} \cdot 3^0 
        &= 21_{(3)} \\
        &n = 8 = \textbf{0} \cdot 3^2 + \textbf{2} \cdot 3^1 + \textbf{2} \cdot 3^0 
        &= 22_{(3)} \\
        &n = 9 = \textbf{1} \cdot 3^2 + \textbf{0} \cdot 3^1 + \textbf{0} \cdot 3^0 
        &= 100_{(3)} \\
        &n = 10 = \textbf{1} \cdot 3^2 + \textbf{0} \cdot 3^1 + \textbf{1} \cdot 3^0 
        &= 101_{(3)} \\
    \end{align*}
\end{bsp}
\begin{bsp}
    Basis 10 ist am bekanntesten:\\
    \(354 = \textbf{3} \cdot 10^2 + \textbf{5} \cdot 10^1 + \textbf{4} \cdot 10^0 \) \\
    als Darstellung zur Basis \(10\): \(a_2 = 3, a_1 = 5, a_0 = 4 \) \\
    \(\Rightarrow 354_{(10)} = 354 \) \\
    Wenn \(b = 5 \), dann ist \\
    \(354 = \textbf{2} \cdot 5^3 + \textbf{4} \cdot 5^2 + \textbf{0} \cdot 5^1 + \textbf{4} 
    \cdot 5^0 = 2404_{(5)}\)
\end{bsp}
\begin{bem} %bessere Struktur?
    Wie findet man eine Darstellung:\\
    Finde größten Index \(m\), so dass \(b^m \leq n\).\\
    Der Koeffizient \(a_m\) ist das größte multiplikative vielfache von \(b^m\), 
    sodass \(n - a_m \cdot b^m \geq 0\).\\
    Dann setzt man \(k=n-a_m \cdot b^m\) und dann setzt man den größten Index \(\tilde{m}\), 
    sodass \(b^{\tilde{m}} \leq k\)
\end{bem}
\begin{bsp}
    4-adische Darstellung von 354
    \begin{align*}			
        &4^5 = 1024 > 354 \\
        &4^4 = 256 < 354 \\
        &354 - 256 = 98 \\
        &\Rightarrow 354 -1 \cdot 4^4 > 0 \text{ und } 354 - 2 \cdot 4^4 < 0 \\
        &\Rightarrow a_4 = 1
    \end{align*}
    Finde Index \(\tilde{m}\), so dass \(4^{\tilde{m}} \leq 98\)			
    \begin{align*}			
        &4^3 = 64 \rightarrow \tilde{m} = 3 \\
        &98 - 64 = 34 < 64 \Rightarrow a_3 = 1 \\
        &4^2 = 16 < 34, 2 \cdot 16 = 32 < 34 \Rightarrow a_2 = 2 \\
        &34 - 2 \cdot 16 = 2 < 4 = 4^1 \Rightarrow a_1 = 0 \\
        &\Rightarrow a_0 = 2
    \end{align*}
    Somit ist die 4-adische Darstellung von \(354\) gegeben durch \(11202_{(4)}\). \\
    \(354 = 111010_{(3)} = 101100010_{(2)} = 542_{(8)}\) \\
    Bleibt zu klären, ob das immer geht.
\end{bsp}
\begin{satz}
    Sei \(b\) eine natürliche Zahl größer als 1. \\
    Jede Zahl hat eine eindeutige Darstellung zur Basis b.
\end{satz}
\begin{bew}
    Konstruiere b-adische Darstellung durch \\
    vollständige Induktion nach n. \\
    (IA):\\
    \(n = 0\) hat die Darstellung \(a_0 = 1 \) \\
    (IS): \(n - 1 \rightarrow n\): sei \(n > 1 \) \\
    Wir nehmen an, dass wir bereits ein b-adische Darstellung für \(n = 1 \) konstruiert haben.
    \[n-1 = \sum_{j=0}^{m} a_jb^j \quad \text{mit } a_m \neq 0 \]
    Wenn \(a_m = a_{m-1} = \dots = a_0 = b-1\), dann stellen wir n als \\
    \(a_{m+1} = 1, a_j = 0\) für \(j \leq m \) \\
    Dies funktioniert, das aus der geometrischen Summenformel folgt:
    \[n-1 = \sum_{j=0}^{m} (b-1)b^j = (b-1) \cdot \sum_{j=0}^{m}b^j = (b-1) 
    \cdot \frac{1- b^{m+1}}{1-b} \]
    \[= \frac{-(1-b)(1-b^{m+1})}{1-b} = b^{m+1}-1 \]
    Ist dies nicht der Fall, muss also mindestens ein Koeffizient in der Darstellung von 
    \(n-1\) kleiner als \(b-1\) sein. \\
    Sei \(\varepsilon\) der kleinste Index, so dass \(a_{\varepsilon} < b-1\). \\
    Wir definieren uns \(c_0, \dots, c_m \) durch: \\
    \(c_j = a_j \quad \text{für } j > \varepsilon \) \\
    \(c_t = a_{t+1}\) \\
    \(c_j = 0 \quad \text{für } j < t\) \\
    Da \(a_j = b-1\) für alle \(j<\varepsilon\) erhalten wir
    \[\sum_{j=0}^{m} c_jb^j = \underbrace{\sum_{j=0}^{t-1} c_jb^j}_{=0} 
    + \underbrace{c_tb^t}_{=a_{t+1}b^t} + 
    \underbrace{\sum_{j=t+1}^{m} a_jb^j}_{=\sum_{j=t+1}^{m} a_jb^j} \]
    Also:
    \[\sum_{j=0}^{t-1} a_jb^j = b^t-1 \]
    \[= 1+\sum_{j=0}^{t-1} a_jb^j + a_tb^t + \sum_{t+1}^{m}a_jb^j 
    = 1+\sum_{j=0}^{m} a_jb^j = n \]
    Nach dem Induktionsprinzip gilt dies somit für alle \(n \in \N \). Wir müssen noch 
    Eindeutigkeit zeigen. \\
    Sei \(a_v,\dots, a_0 \) und \(c_s,\dots, c_0 \) verschiedene Darstellungen von 
    \(n \in \N \). 
    Wenn \(v \neq s \), können o.B.d.A annehmen, dass \(v > s\). Nun ist die durch 
    \(a_v,\dots, a_0 \) dargestellt Zahl mindestens \(b^v \). \\
    Aber die Zahl, die durch \(c_s,\dots, c_0\) dargestellt wird, ist maximal
    \[\sum_{j=0}^{v-1}(b-1)b^j = b^v - 1\]
    Somit kann der Fall nicht einsetzen. Sei also \(v=s\). Wir wissen, dass \(a_v\) und 
    \(c_s\) ungleich Null sind. Somit gilt: \(a_v - 1 \geq 0, c_s - 1 \geq 0\) und wir 
    erhalten somit verschiedene Darstellungen von Basis \(b\) der kleineren Zahl \(n-b^v\). 
    Dies können wir fortsetzen und sind entweder irgendwann in dem Fall, dass die Indizes 
    nicht übereinstimmen (was nicht sein kann) oder die Darstellungen sind schon gleich.
\end{bew}
\begin{bem}
    Sei \(b \in \N, b \geq 2, a \in [0, 1) \)
    \[\alpha = \limes{n} \sum_{j = 1}^{n} \frac{b_j}{b_j} 
    \quad \text{mit } b_j \in \{0,1,\dots,b-1\} \]
    so dass \(l_{n+1} := \lfloor b^{n+1}(x-\sum_{j=1}^{n} \frac{b_j}{b_j})\rfloor\) (\(*\)) \\
    \(b=3 \)
    \begin{center}
    \begin{tikzpicture}[scale = 4]
        \draw (-1/8,0) -- (9/8,0);
        \draw (0,0) node {\([\)} (1,0) node {\()\)};
        \draw (0,0) node[below=3mm] {\(0\)};
        \draw (1,0) node[below=3mm] {\(1\)};
        \foreach \x in {1,2,4,5,7,8} {
        \draw (\x/9,-1/32) -- (\x/9,1/32);}
        \foreach \x in {3,6} {
        \draw (\x/9,-1/16) -- (\x/9,1/16);}
    \end{tikzpicture}
    \end{center}
    ohne (\(*\)) wäre \(\frac{1}{3} = 0,100000\dots = 0,0222\dots\)
\end{bem}

\end{document}