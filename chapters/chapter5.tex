\documentclass[../ana1.tex]{subfiles}
\begin{document}
\setcounter{section}{4}
%09.11.2018
\section{Induktion}
\subsection{Starke Induktion und das Wohlordnungsprinzip}
\begin{satz}[starke Induktion]
	Seien \(A(n)\) Aussagen für \(n\in\N \). Dann gilt
	\begin{enumerate}
		\item \(A(1)\) ist wahr
		\item \(\forall n\in\N: A(1), \ldots, A(n)\) wahr \(\Rightarrow A(n+1)\) ist wahr
	\end{enumerate}
	\(\Rightarrow \forall n\in\N \) ist \(A(n)\) wahr
\end{satz}
\begin{bew}
	Definiere die Aussage \(B(n) := \{ \)alle \(A(k)\) mit \(k\leq n\) sind wahr\( \}\Rightarrow \)
	\begin{enumerate}
		\item \(B(1)\) ist wahr
		\item Ist \(B(n)\) wahr für ein \(n\in\N \), so ist \(B(n+1)\) wahr
	\end{enumerate}
	\(\Rightarrow B(n)\) ist wahr für alle \(n\in\N \).
\end{bew}
\begin{bem}
	\((\forall n\in\N:A(k) \forall k<n \Rightarrow A(n)) \Leftrightarrow \forall n\in\N A(n)\).
\end{bem}
\begin{satz}[Wohlordnungsprinzip für \(\N \)]
	Jede nichtleere Teilmenge der natürlichen Zahlen \(\N \) hat ein kleinstes Element.
\end{satz}
\begin{bew}
	Sei \(A(n):= \{ \)Jede Teilmenge \(b\subset\N \) mit \(m\in B\) hat ein kleinstes Element\( \} \).\\
	Müssen zeigen: \(A(n)\) ist wahr für alle \(n\in\N \).
	\begin{enumerate}
		\item \(A(1)\) ist wahr, denn ist \(B\subset \N \) mit \(1\in B\), so folgt \(\forall k \in B: l\geq 1\). Also ist \(1\) kleinstes Element in \(B\).
		\item Angenommen für \(n\in\N \) sind \(A(1),\ldots,A(n)\) wahr. Sei \(B\subset \N \) mit \(n+1\in B\).\\
		      \underline{1. Fall:} \( \{1,\ldots,n\}\cap B = \emptyset \Rightarrow n+1\) ist kleinstes Element in \(B\).\\
		      \underline{2. Fall:} \( \{1,\ldots,n\} \cap B \neq \emptyset \Rightarrow \exists k\in \{1,\ldots,n\} \) mit \(k\in B\).\\
		      Aus der Induktionsannahme folgt also \(A(k)\) ist wahr. \(\Rightarrow B\) hat ein kleinstes Element.
	\end{enumerate}
	In beiden Fällen hat \(B\) ein kleinstes Element, also ist \(A(n+1)\) wahr.\\
	\(\overset{\text{Satz 1}}{\Rightarrow} \forall n\in \N A(n)\) wahr.
\end{bew}
Notation:\\
Ganze Zahlen \(\Z := (-\N)\cup \N_0 = \{0, \pm 1, \pm 2, \ldots \} = \{\ldots, -2,-1,0,1,2,\ldots \} \).\\
Rationale Zahlen: \(\Q := \{ \frac{m}{n} \vert n\in\N, m\in\Z \} \).
\begin{kor}
	Jede nichtleere, nach unten beschränkte Teilmenge in \(\Z \) hat ein kleinstes Element.
\end{kor}
\begin{bew}
	Sei \(A\subsetneq \Z, A\neq \emptyset, A\geq \beta \) für \(B\in\Z \) \\
	Setze \(B:= A+\beta +1 =\{\alpha+|\beta|+1\vert \alpha\in A\}\subsetneq \N, B\neq \emptyset \\
		\overset{\text{Satz 2}}{\Rightarrow} \exists n_0:= \min B \Rightarrow z_0 := n_0 - |\beta|-1\in\Z \) ist kleinstes Element von \(A\).
\end{bew}
\subsection{Anwendungen}
\begin{lem}
	Sei \( a\in\R \) mit \( a > 0 \). Dann existiert \(q\in\N_0\) mit \(q\leq a<q+1\)
\end{lem}
\begin{bew}
	Ist \( 0 < a < 1 \), so nehme \( q = 0 \).\\
	Also \(a\geq 1\) und setze \(B := \{ n\in\N | a < n \} \).\\
	Da \( \N \) nicht nach oben beschränkt ist (archim. Prinzip), gilt \(B\neq \emptyset \).\\
	\(\overset{\text{Satz 2}}{\Rightarrow} m:=\min B\) existiert. Da \(m\in B\), ist \(m> a \geq 1\).\\
	Somit gilt nach Satz 3.5.8, dass \(q:= m-1\in\N \).\\
	Da \( m \) die kleinste natürliche Zahl mit \(m<a\) ist, folgt \(q = m-1 \leq a < m = q+1\).
\end{bew}
\begin{bem}
	Dieselbe Beweisidee zeigt auch 
	\[ \forall a\in\R\exists q\in\Z \text{ mit } q\leq a<q+1.\]
\end{bem}
\begin{satz}[\( \Q \) ist dicht in \( \R \)]
	Seien \( a,b\in\R, a<b \). Dann existiert \( r\in\Q \) mit \(a<r<b\).
\end{satz}
\begin{bew}
	O.\ B.\ d.\ A.\  \(b\geq 0\), ansonsten betrachte \( a' = -a, b' = -b \).\\
	Weiter können wir \(a\geq 0\) annehmen, sonst nehme \(r=0\).
	Also sei \(0\leq a <b \overset{\text{Archimedes}}{\Rightarrow} \exists n\in\N: n(b-a)>1\).\\
	Setze \( B := \{ l\in\N | l > na \} \subset \N \).
	\[ \overset{\text{Satz 5.1.2}}{\Rightarrow} m = \min B \text{ existiert}.\]
	Da \(m=\min B\) ist, gilt 
	\[m -a\leq na<m,\] 
	somit gilt auch 
	\[ na<m=\underbrace{m-1}_{<na}+\underbrace{1}_{<n(b-a)}=nb \]
	\[ \Rightarrow na<m,nb \Leftrightarrow a<\frac{m}{n}<b.\]
\end{bew}
\textbf{Exkurs}\\
Beh.: \( \sqrt{2}\in\R\setminus \Q \).
\begin{bew}[Beweis durch Widerspruch]
	Sei \( r^2 = 2 \) mit \( r = \frac{m}{n}, n\in\N, m\in\Z \).\\
	Wir definieren 
	\[ A:= \{n\in\N|\exists m\in\Z \frac{m^2}{n^2}= 2\} \neq \emptyset \]
	\[ \overset{\text{Satz 5.1.2}}{\Rightarrow} n_* = \min A \in \N \]
	Also existiert \( m\in\Z_+ \) mit
	\[ m^2 = 2\cdot m_*^2 \Rightarrow m>n_* \]
	Außerdem gilt
	\[ m=\sqrt{2}n_* \overset{\sqrt{2}>1}{\Leftrightarrow} 0< \underbrace{m-n_*}_{\in\N} = \underbrace{\overbrace{(\sqrt{2} - 1)}}^{>0}_{<1} n_* < n_* \]
	Nun gilt: 
	\[ \sqrt{2} = \frac{m}{n_*} = \frac{m(m-n_*)}{n_*(m-n_*)} \overset{m^2=2n_*^2}{=} \frac{2n_*^2-mn_*}{n_*(m-n_*)} = \frac{2n_*-m}{m-n_*} \]
	\Lightning{} \(2n_* -m \in \Z, m-n_* < n_* \), aber \(n_* = \min A\) \\
	Somit kann kein \(m\in\Z \) existieren, sodass \(\frac{m^2}{n^2} = 2\) für beliebiges \(n\in\N \).\\
	Also ist \(\sqrt{2}\) per Definition der rationalen Zahlen in \(\R\setminus\Q \).
\end{bew}
\begin{satz}
	Sei \(k\in\N \), dann gilt entweder \(\sqrt{k}\in\N \) oder \(\sqrt{k}\in\R\setminus\Q \).
\end{satz}
\begin{bew}
	Sei \(k\in\N \) und \(\sqrt{k} \notin \N \).\\
	Angenommen \(\sqrt{k}\in\Q \), also \(\sqrt{k} = \frac{m}{n}, m\in\Z,n\in\N \) \\
	\(A:=\{n\in\N|\exists m\in\Z \frac{m^2}{n^2} = k\} \)
	\[ \overset{\text{Satz 5.1.2}}{\Rightarrow}\exists n_* = \min A \in \N \]
	Sei \(\frac{m}{n_*} = \sqrt{k}\), dann gilt
	\[ m-n_* = \underbrace{(\sqrt{k}-1)}_{<1} n_* \] %DURCHGESTRICHEN
	Aber wähle \(q\in\N: q\leq \sqrt{k} < q+1\) \\
	Existiert nach Lemma 5.2.1. Da \(\sqrt{k} \notin \N \) gilt \(q<\sqrt{k}<q+1\).\\
	Also gilt:
	\[ 0\overset{q<\sqrt{k}}{<} \underbrace{m-qn_*}_{\in\N} = (\underbrace{\sqrt{k}-q}_{<1})n_* < n_* \]
	Somit 
	\[ \sqrt{k} = \frac{m}{n_*} = \frac{m(m-qn_*)}{n_*(m-qn_*)} = \frac{kn_*^2 - mqn_*}{n_*(m-qn_*)}=\frac{kn_*-mq}{m-qn_*}\]
	\Lightning{} \(n_* = \min A, m-qn_* < n_*\) \\
	Somit muss \(\sqrt{k}\in\R\setminus\Q \) sein.
\end{bew}

\end{document}