\documentclass[12pt,a4paper,titlepage]{article} %twopage

%Befehle

%Grafik einbinden
%\centering
%\includegraphics[width=0.7\textwidth]{Bild.pdf}


%\usepackage{german} %deutsches Format
\usepackage[ngerman]{babel}
\usepackage[utf8]{inputenc} %Umlaute
\usepackage{graphicx} %Grafiken einbinden
\usepackage{amsmath,amssymb,amsthm}
\usepackage{nicefrac}
\usepackage{marvosym}[Lightning]
\usepackage{array}
\usepackage{dsfont} %statt \mathbb{1}
\usepackage{tikz}
\usepackage{float}
\usetikzlibrary{matrix}
\usepackage{hyperref} %Hyperlinks in pdf

\restylefloat{table}

%Metadaten
\hypersetup{
	pdftitle = {Analysis I Übungs Skript (WS 18/19)},
	pdfauthor = {Pavel Zwerschke, Daniel Augustin}}

\theoremstyle{definition}
\newtheorem{satz}{Satz}[subsection]
\newtheorem{kor}[satz]{Korollar}
\newtheorem{lem}[satz]{Lemma}

\newtheorem{defi}[satz]{Definition}
\newtheorem*{beh}{Behauptung}

\theoremstyle{remark}
\newtheorem*{bem}{Bemerkung}
\newtheorem*{bsp}{Beispiel}

\newenvironment{bew}{\begin{proof}[Beweis]}{\end{proof}}
\newcommand{\N}{\mathbb{N}}
\newcommand{\Z}{\mathbb{Z}}
\newcommand{\Q}{\mathbb{Q}}
\newcommand{\R}{\mathbb{R}}
\newcommand{\gqq}[1]{\glqq{}#1\grqq{}}
\newcommand{\limes}[1]{\lim\limits_{#1\rightarrow\infty}}

\begin{document}
	\title{Analysis I Übung (WS 18/19)}
	\date{\today}
	\author{Pavel Zwerschke, Daniel Augustin}
	\maketitle
	
	%Inhaltsverzeichnis
	\tableofcontents
	\newpage
	
	\section{Aussagenlogik (19.10.18)}
	\begin{defi}
		Eine Aussage ist die gedankliche Widerspiegelung eines Sachverhalts in Form eines Satzes in einer natürlichen oder künstlichen Sprache.\\\\
		Jede Aussage ist entweder wahr oder falsch.
	\end{defi}
	\begin{bsp}
		Aussagen:
		\begin{enumerate}
			\item Wenn es regnet, dann ist die Straße nass.
			\item Vögel können fliegen
		\end{enumerate}
	\end{bsp}

	\textbf{Aussagen verknüpfen:}
	\begin{align*}
		\neg &A\\
		A &\wedge B\\
		A &\vee B\\
		A &\rightarrow B\\
		A &\Leftrightarrow B
	\end{align*}

	\textbf{Wahrheitstafel:}
	Seien A und B Variablen die entweder wahr oder falsch sind.
	
	\begin{table}[H]
		\centering
		\begin{tabular}{c | c}			
			A & \(\neg A\) \\ 
			\hline
			true & false \\
			false & true
		\end{tabular}
		\caption{Negation}
	\end{table}
	
	\begin{table}[H]
		\centering
		\begin{tabular}{c | c | c | c | c | c}			
			A & B & \(A \wedge B\) & \(A \vee B\) & \(A \rightarrow B\) & \(A \Leftrightarrow B\) \\
			\hline
			w & w & w & w & w & w \\
			w & f & f & w & f & f \\
			f & w & f & w & w & f \\
			f & f & f & f & w & w
			
		\end{tabular}
		\caption{Konjunktion, Disjunktion, Implikation, Äquivalenz}
	\end{table}

	\textbf{Ausdrücken der Aussagenlogik:}
	\begin{enumerate}
		\item Konstanten und Variablen sind Ausdrücke
		\item Sind A und B Variablen, so sind auch (siehe Wahrheitstabelle) Ausdrücke
	\end{enumerate}

	\begin{defi}
		Zwei Aussagenlogische Ausdrücke heißen logisch äquivalent oder wertgleich (A = B), wenn sie die gleichen Wahrheitswerte besitzen.
	\end{defi}

	\begin{bsp}
		\begin{align*}
			(A \wedge B) \vee C &= (A \vee C) \wedge (B \vee C)\\
			\Leftrightarrow (A \wedge B) \vee C &= (B \wedge (A \vee C)) \vee (C \wedge (A \vee C)))\\
			\Leftrightarrow (A \wedge B) \vee C &= (B \wedge A) \vee ((B \wedge C) \vee C)\\
	 		\Leftrightarrow (A \wedge B) \vee C &= (A \wedge B) \vee C
		\end{align*}		
	\end{bsp}

	\begin{bsp}
		\begin{align*}
			\neg(A \wedge B) &= (\neg A \vee \neg B)\\
			(\neg A \vee \neg B) &= (\neg A \vee \neg B)
		\end{align*}
		Somit sind \(\neg(A \wedge B)\) und \((\neg A \vee \neg B)\) logisch äquivalent und es gilt:
		\begin{align*}
			\neg(A \wedge B) = (\neg A \vee \neg B)
		\end{align*}
	\end{bsp}

	\subsection{Tautologie}

	\begin{defi}
		Ein Aussagenlogischer Ausdruck heißt allgemeingültig oder Tautologie, wenn die Wahrheitsfunktion identisch true ist.
	\end{defi}
	
	\begin{bsp}
		Kontraposition:\\
		\begin{align*}
			A \rightarrow B \Leftrightarrow \neg B \rightarrow \neg A
		\end{align*}
		\begin{align*}
			((A \rightarrow B) \rightarrow (\neg B \rightarrow \neg A)) &\wedge ((\neg B \rightarrow \neg A) \rightarrow (A \rightarrow B))\\
			((\neg A \vee B) \rightarrow (B \vee \neg A)) &\wedge ((B \vee \neg A) \rightarrow (\neg A \vee B))\\
			(\neg (\neg A \vee B) \vee (B \vee \neg A)) &\wedge (\neg (B \vee \neg A) \vee (\neg A \vee B))\\
			((A \wedge \neg B) \vee (\neg A \vee B)) &\wedge ((A \wedge \neg B) \vee (\neg A \vee B))\\
			((A \vee (\neg A \vee B)) \wedge (\neg B \vee (\neg A \vee B))) &\wedge ((A \vee (\neg A \vee B)) \wedge (\neg B \vee (\neg A \vee B)))\\
			true &\wedge true\\
			&true
		\end{align*}
	\end{bsp}
	
\end{document}